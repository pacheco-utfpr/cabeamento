%%% LaTeX Template: Two column article
%%%
%%% Source: http://www.howtotex.com/
%%% Feel free to distribute this template, but please keep to referal to http://www.howtotex.com/ here.
%%% Date: February 2011

%%% Preamble
\documentclass[	DIV=calc,%
							paper=a4,%
							fontsize=12pt,%
							onecolumn]{scrartcl}	 					% KOMA-article class

\usepackage{lipsum}													% Package to create dummy text
\usepackage[brazil]{babel}										% English language/hyphenation
\usepackage[protrusion=true,expansion=true]{microtype}				% Better typography
\usepackage{amsmath,amsfonts,amsthm}					% Math packages
\usepackage[pdftex]{graphicx}									% Enable pdflatex
\usepackage[svgnames]{xcolor}									% Enabling colors by their 'svgnames'
\usepackage[hang, small,labelfont=bf,up,textfont=it,up]{caption}	% Custom captions under/above floats
\usepackage{epstopdf}												% Converts .eps to .pdf
\usepackage{subfig}													% Subfigures
\usepackage{booktabs}												% Nicer tables
\usepackage{fix-cm}													% Custom fontsizes
\usepackage[utf8]{inputenc}
\usepackage[top=2.5cm, bottom=2.5cm, left=2.5cm, right=2.5cm]{geometry}
\usepackage[ddmmyyyy]{datetime}
\addto\captionsenglish{%
	\renewcommand\tablename{Tabela}
	\renewcommand\figurename{Figura}
} 
 

 
%%% Custom sectioning (sectsty package)
\usepackage{sectsty}													% Custom sectioning (see below)
\allsectionsfont{%															% Change font of al section commands
	\usefont{OT1}{phv}{b}{n}%										% bch-b-n: CharterBT-Bold font
	}

\sectionfont{%																% Change font of \section command
	\usefont{OT1}{phv}{b}{n}%										% bch-b-n: CharterBT-Bold font
	}



%%% Headers and footers
\usepackage{fancyhdr}												% Needed to define custom headers/footers
	\pagestyle{fancy}														% Enabling the custom headers/footers
\usepackage{lastpage}	

% Header (empty)
\lhead{}
\chead{}
\rhead{}
% Footer (you may change this to your own needs)

%% ====================================
%% ====================================
%% mude o rodape  do projeto
%% ====================================
%% ====================================

\lfoot{\footnotesize \texttt{Cabeamento estruturado} \textbullet ~Projeto de Infraestrutura de Redes - APS Engenharia}


\cfoot{}
\rfoot{\footnotesize página \thepage\ de \pageref{LastPage}}	% "Page 1 of 2"
\renewcommand{\headrulewidth}{0.0pt}
\renewcommand{\footrulewidth}{0.4pt}



%%% Creating an initial of the very first character of the content
\usepackage{lettrine}
\newcommand{\initial}[1]{%
     \lettrine[lines=3,lhang=0.3,nindent=0em]{
     				\color{DarkGoldenrod}
     				{\textsf{#1}}}{}}



%%% Title, author and date metadata
\usepackage{titling}															% For custom titles

\newcommand{\HorRule}{\color{DarkGoldenrod}%			% Creating a horizontal rule
									  	\rule{\linewidth}{1pt}%
										}

\pretitle{\vspace{-10pt} \begin{flushleft} \HorRule 
				\fontsize{40}{40} \usefont{OT1}{phv}{b}{n} \color{DarkRed} \selectfont 
				}

%% ====================================
%% ====================================
%% mude o titulo  do projeto
%% ====================================
%% ====================================

\title{Projeto de Infraestrutura de Rede de Dados - APS Engenharia}					% Title of your article goes here

%% ====================================



\posttitle{\par\end{flushleft}\vskip 0.5em}

\preauthor{\begin{flushleft}
					\large \lineskip 0.5em \usefont{OT1}{phv}{b}{sl} \color{DarkRed}}
\author{Anderson Pacheco dos Santos }  	% Author name goes here


\postauthor{\footnotesize \usefont{OT1}{phv}{m}{sl} \color{Black} 
					\\Universidade Tecnológica Federal do Paraná - Câmpus Cornélio Procópio 								% Institution of author
					\par\end{flushleft}\HorRule}

\date{}																				% No date




%%% Begin document
\begin{document}
\maketitle
\thispagestyle{fancy} 	
\thispagestyle{empty}		% Enabling the custom headers/footers for the first page 
% The first character should be within \initial{}




%% ====================================
%% ====================================
%% mude o resumo  do projeto
%% ====================================
%% ====================================

\initial{O}\textbf{presente projeto destina-se ao planejamento da implantação simulada da infraestrutura da rede de comunicação de dados da empresa fictícia APS Engenharia. Para tanto, tem-se a premissa de que o edifício encontra-se incorporado, no entanto, não possui elementos de infraestrutura adequados para a instalação de componentes de rede. O presente estudo irá definir um novo projeto de redes para a empresa, observando os passivos que serão empregados, a fim de garantir o plano de certificação. Com o levantamento da planta física da edificação, será executado o levando de custos e quantidade de materiais, bem como da terceirização de serviços de instalação de condutores externos, haja vista o prédio não possuir tal elemento de infraestrutura. Constituirá o corrente documento, informações sobre os procedimentos de manutenção após a implantação. Registra-se ainda que a empresa torna-se diretamente dependente da rede de comunicação de dados, uma vez que, todas as ferramentas de software e armazenamento de dados da empresa estão diretamente ligados aos serviços de cloud computing.}


%% ====================================
\begin{figure}
	\centering
	\includegraphics{utfpr}
\end{figure}

\vspace{2cm}
\centerline{\textit{\textbf{\today}}}

\clearpage
    \renewcommand*\listfigurename{Lista de figuras}
\listoffigures

\renewcommand*\listtablename{Lista de tabelas}
\listoftables




\clearpage
\renewcommand{\contentsname}{Sumário}
\tableofcontents
\clearpage

%% ====================================
%% ====================================
%% Inicio do texto
%% ====================================
%% ====================================
\section{Introdução}
A empresa fictícia APS Engenharia atua no setor de prestação de serviços de telecomunicação, possuindo contrato ativo de outsourcing com uma organização de grande porte de referência nacional no segmento. A empresa em estudo é considerada de médio porte e possui em seu quadro funcional diversos colaboradores, no entanto, levando em consideração as necessidades do delineamento do presente documento, serão considerados apenas os funcionários que desenvolvem suas atividades na sede da empresa, local pelo qual está sendo formulado o presente projeto.
O parque de informática da organização é moderno, sendo os dispositivos adquiridos recentemente, estando ainda no período de garantia do fabricante. A conexão com a rede mundial de computadores é fornecida através de provedor via rádiofrequêcia em velocidade contratual de 60 Mbps, com transmissão de 2.4 Ghz. 
O propósito do referido estudo é subsidiar a equipe técnica e de gestão da empresa na preparação do edifício para implantação de rede de dados cabeada, capaz de atender as demandas administrativas que atualmente possuem sistemas de software e armazenamento de dados em nuvem, com projeção de certificação da rede e contratação de plano de provimento de comunicação de dados dedicado acima de 200 Mbps, fornecidos pela operadora através de rede de fibra óptica.


\subsection{Benefícios}
Dentre os diversos benefícios de um projeto de estruturação de rede de dados está a garantia de usabilidade e de disponibilidade da rede, a possibilidade de certificação física da estrutura de passivos, ressaltando que a atividade fim da organização é justamente a prestação de serviços no segmento de comunicação. Alguns benefícios secundários e não menos importantes estão relacionados a qualidade técnica para desenvolvimento das atividades administrativas e na agilidade e facilidade de prestar suporte técnico preventivo e corretivo na infraestrutura de dados.

\subsection{Organizações Envolvidas}
Para a implantação do projeto, algumas empresas/profissionais deverão ser contratadas para a realização dos serviços de acesso, infraestrutura, suporte e manutenção. 

\begin{table}[]
	\centering
	\caption{Empresas envolvidas no projeto}
	\label{tab_empresas}
	\begin{tabular}{|l|l|}
		\hline
		\multicolumn{1}{|c|}{\textbf{Descrição  do Serviço}} & \multicolumn{1}{c|}{\textbf{Empresa  / Profissional}} \\ \hline
		Projeto de rede                                      & Engenheiro de telecomunicação                         \\ \hline
		Instalação de eletrodutos                            & Engenheiro civil / Engenheiro elétrico                \\ \hline
		Instalação da rede cabeada                           & Instalador de redes                                   \\ \hline
		Orçamento e compras                                  & Analista de compras                                   \\ \hline
		Provedor de internet – Acesso Principal              & Empresa de Telecom 1                                  \\ \hline
		Provedor de internet – Acesso de contingência        & Empresa de Telecom 2                                  \\ \hline
		Certificação da rede                                 & Empresa certificadora credenciada                     \\ \hline
		Instalação e suporte dos equipamentos de informática & Engenheiro da Computação                              \\ \hline
	\end{tabular}
\end{table}



\section{Estado atual}
A rede de comunicação de dados atual não possui padronização, sendo atendida basicamente por um provedor de internet que fornece a conexão via rádio-frequência, e é distribuída internet para as estações de trabalho via pontos de acesso sem fio, com repetidores da mesma natureza ao longo do prédio.

\begin{itemize}
	\item Passivos de rede atuais:2 roteadores sem fio e 4 repetidores sem fio;
	\item Principais reclamações dos usuários: baixa qualidade de transmissão de dados, indisponibilidade recorrente da rede, falta de contingência de dados e provedor de serviços de internet não oferece suporte imediato.
	\item Observações: O estado atual da rede é de extrema vulnerabilidade, uma vez que os pontos de acesso não estão ocultos e podem ser utilizados por pessoas mal intencionadas. Não existe um padrão ou norma que justifique o estado atual da rede, sendo que foi instalada emergencialmente apenas para atender as demandas administrativas da empresa até o término e implantação do projeto de rede cabeada.
\end{itemize}

\section{Requisitos}
Crie uma enumeração dos requisitos do projeto.

\section{Usuários e Aplicativos}

\subsection{Usuários}
A empresa possui 21 colaboradores, sendo que 19 são usuários ativos do parque de informática da entidade.
Os usuários em sua maioria exercem funções exclusivamente administrativas e em geral possuem conhecimentos amplos para utilização dos recursos tecnológicos disponíveis, sendo atendidos por um colaborador que exerce a função de suporte e é responsável pelos ativos e passivos da estrutura tecnológica da empresa. Este por sua vez, possui formação acadêmica na área de ciências da computação.
Existe a perspectiva de crescimento de 20 porcento no quadro de colaboradores para os próximos 3 anos.

\subsection{Aplicativos}
As ferramentas e serviços utilizados na unidade encontram-se em sua maioria na nuvem, tornando esse o principal e mais importante ponto crítico a ser considerado nesse projeto. Os usuários utilizam o Domínio de rede para login nas estações de trabalho o qual é validado em LDAP remoto, utilizam sistemas de Intranet, provedor de e-mail próprio, ferramentas e softwares remotos, armazenamento de arquivos em servidor externo (NAS), acesso a internet com permissões de servidores de proxy e firewall remotos, entre outros.


\section{Estrutura predial existente}


%inicio dos comandos para criar uma nova pagina A3 vertical
\clearpage
\KOMAoptions{paper=a3, paper=portrait, DIV=15}
\recalctypearea


\begin{figure}
	%	\centering
	\noindent\makebox[\textwidth][c]{
		\includegraphics[height=\textheight]{plantabaixa}
	}
	\caption{Planta Baixa do Edifício da Empresa APS Engenharia.}
	\label{plantabaixa}
\end{figure}

%Retornar ao formato A4
\clearpage
\KOMAoptions{paper=a4, paper=portrait, DIV=15}
\recalctypearea
%-- reinicio em A4 


\section{Planta Lógica - Elementos estruturados}

\subsection{Estado atual}


%inicio dos comandos para criar uma nova pagina A3 vertical
\clearpage
\KOMAoptions{paper=a3, paper=portrait, DIV=15}
\recalctypearea


\begin{figure}
	%	\centering
	\noindent\makebox[\textwidth][c]{
		\includegraphics[height=\textheight]{plantalogica.png}
	}
	\caption{Planta Lógica do Edifício da Empresa APS Engenharia.}
	\label{plantalogica}
\end{figure}

%Retornar ao formato A4
\clearpage
\KOMAoptions{paper=a4, paper=portrait, DIV=15}
\recalctypearea
%-- reinicio em A4 


\subsection{Topologia}
Proposta futura, proposta após implantação.
Deve conter o diagrama da rede. Atente-se a redundância  e ligações truncadas.
Deve explicar todos termos e componentes utilizados nestas plantas. Por exemplo: entrance facility, work area, horizontal cabling, etc..

Todos os elementos das figuras devem ser explicados. 
Crie esboço da configuração dos racks e brackets. Explique cada um dos componentes. Você pode criar uma tabela contendo figuras dentro, ou criar uma tabela e incluí-la como imagem. Por exemplo, verifique a tabela \ref{tab1}.

\input{tab1}

\subsection{Encaminhamento}
Eletrodutos, calhas, e qualquer material em que os cabos serão alojados/alocados.

\subsection{Memorial descritivo}
\begin{table}[]
	\centering
	\caption{Descrição dos elementos passivos do projeto}
	\label{tab_passivos}
	\begin{tabular}{|l|c|c|c|}
		\hline
		\textbf{ELEMENTO PASSIVO}  & \textbf{QUANTIDADE} & \textbf{MEDIDA} & \textbf{FABRICANTE} \\ \hline
		Rack Padrão Piso 32u       & 1                   & un              & Saint Blanc         \\ \hline
		Patch Panel 24 Portas Cat6 & 2                   & un              & Furukawa            \\ \hline
		Patch Cords Cat 6 - 1m     & 20                  & un              & Furukawa            \\ \hline
		Patch Cords Cat 6 - 2m     & 20                  & un              & Furukawa            \\ \hline
		Cabo UTP Cat 6             & 966                 & m               & Furukawa            \\ \hline
		Tomadas para embutir RJ45  & 40                  & un              & Enerbras            \\ \hline
		Eletroduto 2"              & 98,5                & m               & Tigre               \\ \hline
		Eletroduto 1.1/4"          & 100                 & m               & Tigre               \\ \hline
		Fita Hellerman 230mm       & 200                 & un              & Tyton               \\ \hline
		Régua de energia 1u        & 2                   & un              & Fiolux              \\ \hline
		Régua organizadora 1u      & 2                   & un              & Pier Telecom        \\ \hline
		Bandejas para Rack 2u      & 3                   & un              & Cwb Metal           \\ \hline
	\end{tabular}
\end{table}
Relacione todos os equipamentos passivos que serão utilizados, tipo, fabricante, quantidade.



\subsection{Identificação dos cabos}
Explique como os cabos serão identificados em seu projeto. Coloque uma relação dos cabos instalados e identificados.

\section{Implantação}
Estabeleça um cronograma de implantação:
Remoção de equipamentos existentes (destino para descarte), instalação dos condutores, instalação dos cabos, 
identificação dos cabos, montagem dos racks, certificação, etc... Crie atividades e estabeleça o tempo de execução. Se for um projeto real, indique também quais os responsáveis pela execução do projeto e de cada uma das etapas.

Defina marcas (e padrões) e fornecedores se for o caso. Atenção a contratados e subcontratados para a realização das atividades. Estabeleça a responsabilidade de execução da atividade e também da validação dela.

Utilize algum software para gerear o cronograma. Excel,etc. O fundamental é dividir em etapas, descrever e estimar o tempo de cada uma delas.

Segue uma relação de ferramentas:
http://asana.com/, 
https://trello.com/, 
http://www.ganttproject.biz/, 
http://www.orangescrum.org/. 

\section{Plano de certificação}
Quais seriam as etapas para a certificação? 
Quais os locais e horários para execução da certificação na rede? Toda rede será certificada?
Como os testes seriam executados?
Quais relatórios de certificação serão (ou deveriam ser) entregues? 

\section{Plano de manutenção}

Revisões periódicas na rede, emissão de certificados para novos pontos.

\subsection{Plano de expansão}
Existe um plano de expansão? Quantos novos pontos poderão ser acrecidos na rede, antes de migração de equipamentos na camada 2? Se houver expansão, quais equipamentos deverão ser direcionados para as estremidades da rede? 

\section{Risco}
Enumerar e explicar os riscos do projeto.

\section{Orçamento}
Crie uma relação de orçamentos baseado na seções anteriores.

\section{Recomendações}
Observações e recomendações para o cliente.

\section{Referências bibliográficas}
Utilize o mendley, o jabref ou diretamente o bibtex para gerenciar suas referências biliográficas. As referências são criadas automaticamente de acordo com o uso no texto.

Exemplo: Redes de computadores, segundo \cite{t2013} é considerada..... Já \cite{kurose2010} apresenta uma versão...

Analisando os pressupostos de \cite{ref3} e \cite{ref4} concluimos que....


\renewcommand\refname{} %%Referências bibliográficas}  
\bibliographystyle{ieeetr}
\bibliography{referencias}  


\end{document}